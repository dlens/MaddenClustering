\documentclass[11pt]{article}
\usepackage{fontspec}
%\usepackage{xltxtra}

\usepackage[design]{dljmv2overleaf}

\title{Madden Project Phase 2 - Notes}

\author{Gavin Byrnes}
\date{June 9, 2016}

%\setcounter{tocdepth}{2}

\usepackage{hyperref}


\begin{document}
\maketitle

\tableofcontents

\begin{abstract}

\end{abstract}

\section{Introduction}

To test offense we will use a very generic defense.

\section{Teams And Experiments}

So we have the generic defense of 50s - currently with a team built for West Coast...so we run that team with West Coast offense and also with others.

What we probably want to do is make minor tweaks to the rosters positionally speaking

Also probably something about making the defense better?

\section{Plans And Options}

\subsection{Gameplan}

A team is built to fit a specific game plan, with generic defense (or offense). The team is then simulated against itself 100 times using different game plans; the idea here is to demonstrate that the different archetypes are effective at identifying which players are better suited to each other and to different gameplans.
\subsection{Incremental Change}
On Offense, consider: 
QBs
RBs
WR/TEs
O-Line

Rate High, Medium, Low.

There are 81 possible combinations; test all of them with the same defense.

On defense, consider:
D-Line
Linebackers
Secondary

Rate High, Medium, Low.

There are 27 possible combinations; test all of them with the same defense.
\subsection{Team Evaluation}
Teams either randomly or exactly constructed. Two ideas, 1: give them team ratings based on individual grades, run correlations on results using these ratings. 2: run correlations on *all* individual grades.

But for example, we'll have things like: A QB, B RB, C WR/TE, B O-Line, A D-Line, B Linebackers, B Secondary vs. B QB, C RB, B WR/TE, C O-Line, D D-Line, C Linebackers, A Secondary (or whatever) - and we see how that resolved itself.

For the H/M/L or A/B/C/D/F we will need to apply a simpler grading method...wait we just use what we have. It's already on a 0-100 scale of normality, just tweak it a bit. 

something like
80-100: A
60-80: B
40-60: C
20-40: D
0-20: F

or maybe, cause it's just starters:
85-100: A
70-85: B
50-70: C
30-50: D
0-30: F



\bibliographystyle{plain}
\bibliography{programming}

\end{document}